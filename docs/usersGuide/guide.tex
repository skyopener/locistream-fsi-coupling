\documentclass{article}
\usepackage{courier}
\usepackage{times,mathptmx}

\topmargin  0.0in
\headheight  0.15in
\headsep  0.15in
\footskip  0.2in
\textheight 9.45in
                                                                                
\oddsidemargin 0.56in
\evensidemargin \oddsidemargin
\textwidth 5.8in

\pagestyle{myheadings}
\markright{\bf Loci-Stream User's Guide - Release 1.3.9}
                                                                                
\begin{document}

\title{Loci-Stream: A Pressure-Based, Finite-Rate Chemistry Solver for
Combustion Flows, Release 1.4.8 -- Users' Guide}
                                                                                
\author{Siddharth Thakur and Jeff Wright}

\maketitle

\newcommand{\cpp}{$C^{++}$}
\newcommand{\CVODE}{\emph CVODE}
\newcommand{\Loci}{\emph Loci(Version 3.2-beta-12)}
\newcommand{\Stream}{\emph Loci-Stream}
\newcommand{\Hypre}{\emph HYPRE(Version 1.10.0b)}
\newcommand{\Sundials}{\emph Sundials(Version 2.3.0)}
\newcommand{\Chemkin}{\emph Chemkin}
\newcommand{\Chem}{\emph Chem(Version 3.2-beta-12)}
\newcommand{\ChemUsersGuide}{\emph Chem 3: A Finite-Rate Viscous Chemistry Solver - The User Guide}

\section{Installation and Required Software Libraries}

Before attempting to compile \Stream\ one must install (or have installed) the
following software libraries:

\begin{enumerate}
  \item The \Loci\ program development environment.
  \item The \Chem\ code.
  \item The \Hypre\ linear solver library.
  \item The \Sundials\ CVODE library. This can be obtained from
\emph{http://www.llnl.gov/CASC/sundials}.
\end{enumerate}

\Stream\ is distributed via the bzipped tar file stream-rel-1-4-8.tar.bz2 .
After unpacking the
distribution (assuming \emph{\bf /usr/local/}) one should find the
directory \emph{\bf /usr/local/stream-rel-1-4-8}. There are three
sub-directories under the main directory as follows:
\begin{description}
  \item[docs] This directory contains a number of documents decribing the
methodology behind \Stream\ as well as this users' guide.
  \item[examples] This directory contains a number of sample problems which
can be used either to test \Stream\ or used as templates for constructing case
files for new problems.
  \item[src] This directory contains the source code for \Stream\ .
\end{description}
To compile \Stream\ , perform the following:
\begin{enumerate}
  \item Edit the file \emph{\bf Stream.conf} located in the main distribution
directory, and provide the appropriate values for the variables
\emph{STREAM\_INSTALL\_DIR}, \emph{LOCI\_BASE},
\emph{STREAM\_BASE}, \emph{CHEM\_BASE}, \emph{HYPREDIR} , \emph{SUNDIALSDIR}
and \emph{CHEMMODULEDIR}.
  \item Execute the \emph{make} command. Upon successful compilation, one should
find the executables \emph{\bf stream} and \emph{\bf stream-piso} in 
the \emph{\bf bin} subdirectory of the specified install directory.
  \item Add the \Stream\ executable directory \emph{STREAM\_INSTALL\_DIR/bin} to
the PATH variable in your shell start-up file.
\end{enumerate}
To run the SIMPLE-based version of \Stream\, execute the following command:
\begin{verbatim}
  mpirun -np 1 stream -q solution [-inputs] caseName [debug]
    [restartIteration]
\end{verbatim}

To use the PISO-based version of \Stream\, use 'stream-piso' in place of 'stream'.
One can vary the number of processors as desired by modifying the value of
the \emph{-np} option. The optional \emph{-inputs} flag can be used to
examine the appropriate input variables and their default values. When the
optional \emph{debug} flag is used, certain
diagnostic output (such as maximum residual locations for each equation) will
be written to the file \emph{debug/debug}. Include the \emph{restartIteration}
option only when re-starting from a previous run. To determine the restart
iteration numbers available, examine the files in the \emph{\bf restart}
directory. \textbf {Note: If both the debug and restart iteration number options
appear in the command line, they must be in the order specified above.}

\section{Note on Units}

As noted in the following section, while some input to \Stream\ may be
specified in arbitrary units, all input data is converted into the MKS unit
system, in which length is measured in \emph{m}, mass in \emph{kg}, time in
\emph{s}, pressure in \emph{Pa} and temperature in \emph{K}. Upon successful
running of a problem, data in all output files is expressed in these units,
including flow variables and grid coordinates.

\section{Problem Specification}

Three data files are required for running \Stream\ . Assuming a case name
of \emph{caseName}, these files are \emph{caseName.vog}, \emph{caseName.vars},
and a chemistry model file such as H2O2\_6s9r.mdl . If \Chemkin\ transport
properties are required, a fourth file such as H2O2\_6s9r.tran is also needed.
For information on the grid file (.vog file), chemistry model and transport
properties files, see users' guide contained in the \emph{Chem}
distribution. \emph{\bf IMPORTANT: While the old .xdr format is still
supported, one must use the new .vog format for restarts to work properly.}
Standard chemistry model and transport data files from the
\Chem\ distribution can be accessed by defining the \emph{CHEMISTRY\_DATABASE}
enviroment variable as described in the \Chem\ users guide.

Problem specification, including grid information, boundary conditions and
numerical solver data must be specified in the so-called \emph{.vars} file.
The \emph{\bf examples} directory contains a number of cases covering the range
of capability of \Stream\, including incompressible/compressible,
inviscid/viscous and non-combusting/combusting flows. The best way to create a
\emph{.vars} file for a new problem is to copy and modify one of these template
files, choosing one with similar physics to the new problem of interest. A
description of each problem appears at the top of its \emph{.vars} file.

The \emph{.vars} file consists of a number of \emph{facts} describing the
problem, which can be listed in any order. In the following, we describe each
of the possible \emph{facts} that can be present in the \emph{.vars} file, the
use of the \emph{fact} and the conditions under which it is required.

\begin{enumerate}

\item{\tt grid\_file\_info}

This fact is used to describe the grid file and its contents and consists of
an options list defined by a $< >$ enclosed list of options. The
\emph{file\_type} option is required, and the \emph{Lref} option can be
optionally used to specify the reference length for the grid. The only fully
supported \emph{file\_type} is the \emph{VOG} grid type. There is an extensive
number of units supported for the \emph{Lref} option. For example, if the grid
provided is in \emph{VOG} format and has units of inches, the entry would
appear as follows:
  \begin{verbatim} grid_file_info: <file_type=VOG,Lref=1 inch> \end{verbatim}
This fact is required for all cases.

\item{\tt rigidBodyGridMotion}

This fact is used to specify grid motion parameters for problems with rigid-
body grid motion. See \emph{\bf Appendix \ref{appendix:rigidBody}} for details
on the use of this fact.

\item{\tt boundary\_conditions}

This fact is used to associate each tag number in the \emph{VOG} grid file
(which denotes a subset of boundary faces of the domain) with a boundary
condition type, and consists of an options list defined by a $< >$ enclosed
list of boundary condition tag identifiers and their corresponding boundary
condition type. For example, for an incompressible problem with a single inlet,
single outlet, two no-slip walls and two symmetry boundaries, this fact might
appear as follows:
\begin{verbatim}
 boundary_conditions:
 <
  BC_1=symmetry,BC_2=symmetry,
  BC_3=noslip(wallFunction),BC_4=noslip(wallFunction),
  BC_5=incompressibleInlet(rho=998.7,v=2.86808e-02,k=4.5167e-08,
    omega=4.5167),
  BC_18=fixedPressureOutlet(p=0.0)
 >
\end{verbatim}
The number following the marker \emph{BC\_} for each boundary condition
corresponds to the tag number for the subset of boundary faces in the
\emph{.vog} file that defines the boundary. Note that in the \emph{.vog} file,
this tag is actually a negative number (e.g. -1 in the \emph{.vog} file
corresponds to \emph{BC\_1}. In addition, boundary conditions need not be
sequentially numbered and can be placed in any order. Each boundary condition
type consists of a name (e.g. \emph{noslip}) and an optional comma separated,
$()$ enclosed list of options. The following is a list of all currently
supported boundary condition types:
\begin{description}
  \item{\bf symmetry} This type is used for symmetry boundaries.
There are no options required with this type.
  \item{\bf slip} This type is used for slip boundaries. There are no options
required with this type. One should not use both slip and noslip boundaries
in the same simulation. If a slip wall is needed in a problem that contains
noslip boundaries, use symmetry boundaries for the slip surfaces.
  \item{\bf noslip} This type is used for no-slip boundaries. For turbulent flow
cases where wall functions are desired, use the \emph{wallFunction} option.
For walls with specified temperature, use the \emph{T=value} option. For
an adiabatic boundary, use the \emph{adiabatic} option, although this is the
default.
  \item{\bf incompressibleInlet} This is an inlet boundary condition type for
incompressible problems. Density is
specified with the \emph{rho=value} option. One must specify either the mass
flow rate (e.g. \emph{kg/s}) with the \emph{mdot=value} option, the mass flux
(e.g. \emph{$kg/s*m^2$}) , with the \emph{massFlux=value} option or the
velocity with the \emph{v=value} option. If the velocity value is a scalar,
the x-coordinate direction is assumed. For vector specification, use the form
\emph{v=[vX,vY,vZ]}. The flow direction can also be specified with the
\emph{flowDirection=value} option when specifying either the mass flow rate or
mass flux options. The value for this option may be specified as a scalar
(x-coordinate direction flow assumed), or as a vector using the form
\emph{flowDirection=[fX,fY,fZ]}.  If the \emph{flowDirection=value} option is
not present when specifying a mass flow rate or mass flux, the resulting
velocity will be assigned normal to the boundary faces, going into the domain.
For turbulence quantities, use the \emph{k=value} and
\emph{omega=value} options.
  \item{\bf subsonicInlet} This compressible-flow inlet type is for inlet
boundaries with subsonic flow conditions. One must specify either the mass flow
rate (e.g. \emph{kg/s}) with the \emph{mdot=value} option, the mass flux
(e.g. \emph{$kg/s*m^2$}) with the \emph{massFlux=value} option or the velocity
with the \emph{v=value} option. The flow
direction may also be specified with the \emph{flowDirection=value} option in
the same manner as incompressibleInlet. Turbulence quantities
are assigned in the same manner as incompressibleInlet.  For species
mass fractions, use the option
\emph{mixture=[species0=value,species1=value,...]}, where species0,species1,
\ldots\ are the names of the species found in the \emph{.mdl} file. If a
species is not listed, a zero mass fraction is assumed.
  \item{\bf supersonicInlet } This compressible-flow inlet type is for inlet
boundaries with supersonic flow conditions. One must specify the density with
the \emph{rho=value} option, the velocity with the \emph{v=value} option, the
pressure with the \emph{p=value} option, and the temperature with the
\emph{T=value} option. Turbulence quantities and species mass fractions are
assigned in the same manner as described above for
incompressibleInlet.
  \item{\bf totalPressureInlet } This compressible-flow inlet type is for inlet
boundaries with specified total pressure. One must specify the total pressure
with the \emph{p0=value} option. One can specify either the static temperature
with the \emph{T=value} option, or the total temperature with the
\emph{T0=value} option. Turbulence quantities and species mass fractions are
assigned in the same manner as described above for
incompressibleInlet. \textbf {Warning: Ideal-gas assumptions are
made with this boundary condition, so do not use for real-fluid boundaries}.
  \item{\bf fixedPressureOutlet} This outlet boundary condition is appropriate
for both incompressible problems and compressible problems where the flow
remains subsonic at all times on the outlet boundary. One can specify either a
constant outlet pressure using the \emph{p=value} option or the mean outlet
pressure using the \emph{pMean=value} option.
  \item{\bf extrapolatedPressureOutlet} This outlet boundary condition is
appropriate for incompressible problems as well as compressible problems
where the flow is supersonic at the outlet. There are no options required
by this type. For incompressible problems, global mass conservation is
automatically performed.
\end{description}

One may specify boundary profiles for the variables velocity and temperature as
well as the turbulence quantities k and omega by using either the Cartesian or
axisymmetric input file formats described in \emph{\bf Appendix
\ref{appendix:boundaryProfile}} . For
example, to specify a Cartesian velocity profile (a profile varying in either
the x, y or z direction), use the \emph{v=cartesian(file="filename")} option on any
boundary where the \emph{v=value} option is allowed. The input filename can be
arbitrarily selected by the user. To specify an axisymmetric velocity profile,
one would similarly use \emph{v=axisymmetric("filename")}.

\item{\tt initialCondition}

This fact is used to specify initial conditions for the whole domain. For incompressible
flows, one must specify values for density, velocity and pressure (using the
same options as shown above for boundary conditions). For compressible flows,
one must specify the velocity, pressure and temperature. For turbulent flow,
specify \emph{k} and \emph{omega}. For flows with multiple species, specify the
species mass fractions. One must use either this fact,
{\tt initialConditionRegions} or {\tt ql} and {\tt qr} for all cases.

\item{\tt initialConditionRegions}

This fact, described more completely in \emph{\bf Appendix \ref{appendix:initialCondition}}
can be used to more finely tune the initial conditions througout the domain.
NOTE: This option is currently only available for compressible flows.

\item{\tt ql and \tt qr}

These facts are used to specify the starting flow conditions for problems
with an initial left and right state. Options are identical to those for
{\tt initialCondition} . By default, the partition between the left and
right states is \emph{x=0.0} .

\item{\tt xmid}

This fact is used to adjust the partition location between the left and right
states. Specify a real value.

\item{\tt flowRegime}

This fact is used to select either laminar or turbulent flow. Valid values are
\emph{laminar} and \emph{turbulent}. This fact is required for all cases.

\item{\tt flowCompressibility}

This fact is used to select either incompressible or compressible flow. Valid
values are \emph{incompressible} and \emph{compressible}. This fact is required
for all cases.

\item{\tt chemistry\_model}

This fact is used to select the .mdl file. See \ChemUsersGuide\ for a
description of this fact. This fact is required for all cases.

\item{\tt thermodynamic\_model}

This fact is used to specify the thermodynamic model. See \ChemUsersGuide\ for
a description of this fact. This fact is required for all cases.

\item{\tt transport\_model}

This fact is used to specify the transport model for laminar viscosity and
thermal conductivity. See \ChemUsersGuide\ for a description of this fact.
This fact is required for all cases.

\item{\tt Tf}

This fact represents a freezing temperature, below which species source terms
due to combustion cease to be active. If the temperature in a cell is less
than this value, species production due to combustion is deactivated. Setting
a large value (such as 1.0e30) will effectively shut down all chemical
reactions for the entire domain.

\item{\tt turbulentPrandtlNumber}

This fact is used to set the turbulent Prandtl number. The default value
is 0.7 .

\item{\tt laminarSchmidtNumber}

This fact is used to set the laminar Schmidt number for all species.
The default value is 0.9 .

\item{\tt turbulentSchmidtNumber}

This fact is used to set the turbulent Schmidt number for all species.
The default value is 0.95 .

\item{\tt timeIntegrator}

This fact is used to specify the time integrator. Valid values are \emph{BDF},
which is the first-order backward Euler method and \emph{BDF2}, which is the
second-order, three-level backward method. This fact is required for all cases.

\item{\tt timeStep}

This fact is used to specify the time-step size. It is required for all cases.

\item{\tt numTimeSteps}

This fact is used to specify the number of time-steps for the current run.
It is required for all cases.

\item{\tt convergenceTolerance}

This fact is used to specify the convergence tolerance (e.g 1.0e-04) at each
time-step in the calculation. There is a single tolerance for all equations
being solved. When the tolerance is achieved at a time-step, the solver moves
to the next time level.

\item{\tt maxIterationsPerTimeStep}

This fact is used to specify the maximum number of iterations per time-step.

\item{\tt inviscidFlux}

This fact is used to select the inviscid flux type for all equations. Specify
\emph{FOU} for first-order upwinding, \emph{SOU} for second-order upwinding,
or \emph{Roe} for the Roe scheme. The Roe scheme can be used only for
compressible flows.

\item{\tt turbulenceInviscidFlux}

By default this fact assumes the same value as the fact {\tt inviscidFlux}.
One may choose to solve the turbulence equations using first-order upwinding
while using second-order upwinding on the mean-flow equations by setting this
fact to \emph{FOU} .

\item{\tt limiter}

This fact is used to specify the limiter when using the \emph{SOU} option for
the inviscid flux. Valid values are \emph{none}, \emph{venkatakrishnan} and
\emph{barth} . For complex flows, some form of limiting may be required for
stability purposes.

\item{\tt momentumEquationOptions}

This fact is used to specify solution parameters for the momentum equation,
and is required for all cases.  Three options must be specified as follows:
  \begin{enumerate}
     \item{\bf linearSolver} Choose \emph{SGS} for the symmetric Gauss-Seidel
solver.
     \item{\bf relaxationFactor} The relaxation factor used in the non-linear
iterative process. This option is not used by 'stream-piso'.
     \item{\bf maxIterations} The maximum number of iterations to perform in
the linear equation solver.
  \end{enumerate}

\item{\tt pressureEquationOptions}

This fact is used to specify solution parameters for the pressure-correction
equation, and is required for all cases. Three options must be specified as
follows:
  \begin{enumerate}
     \item{\bf linearSolver} Choose \emph{SGS} for the symmetric Gauss-Seidel
solver, \emph{PETSC} for the PETSc Krylov solver or \emph{HYPRE} for the HYPRE
algebraic multigrid solver.
     \item{\bf relaxationFactor} The relaxation factor used in the non-linear
iterative process. This option is not used by 'stream-piso'.
     \item{\bf maxIterations} The maximum number of iterations to perform in
the linear equation solver.
  \end{enumerate}

\item{\tt energyEquationOptions}

This fact is used to specify solution parameters for the energy equation,
and is required for all compressible flow cases.  Three options must be
specified as follows:
  \begin{enumerate}
     \item{\bf linearSolver} Choose \emph{SGS} for the symmetric Gauss-Seidel
solver.
     \item{\bf relaxationFactor} The relaxation factor used in the non-linear
iterative process. This option is not used by 'stream-piso'.
     \item{\bf maxIterations} The maximum number of iterations to perform in
the linear equation solver.
  \end{enumerate}

\item{\tt turbulenceEquationOptions}

This fact is used to specify solution parameters for the turbulence equations,
(both \emph{k} and \emph{omega}) and is required for all cases where
\emph{flowRegime=turbulent}. Four options must be specified as follows:
  \begin{enumerate}
     \item{\bf model} Choose \emph{menterBSL} for Menter's baseline model,
\emph{menterSST} for Menter's original shear-stress transport model or
\emph{menterSST2003} for Menter's updated shear-stress transport model.
     \item{\bf des} Used to specify the Detached-Eddy Simulation model.
Choose \emph{DES} for Strelet's 2001 DES formulation or \emph{DDES} for
Menter's Delayed DES formulation. This is the only optional option.
     \item{\bf linearSolver} Choose \emph{SGS} for the symmetric Gauss-Seidel
solver.
     \item{\bf relaxationFactor} The relaxation factor used in the non-linear
iterative process. This option is not used by 'stream-piso'.
     \item{\bf maxIterations} The maximum number of iterations to perform in
the linear equation solver.
  \end{enumerate}

\item{\tt kFreestream}

This fact is used to assign cell values for turbulent kinetic energy when
negative values have been returned by the PDE. The default value is 1.0e-08 .

\item{\tt omegaFreestream}

This fact is used to assign cell values for turbulent dissipation rate when
negative values have been returned by the PDE. The default value is 10.0 .

\item{\tt turbulenceIntensityFreestream}

This fact is used to assign cell values for turbulent kinetic energy and
dissipation rate when negative values have been returned by the PDE for
either k or omega. When limiting is activated, local values of density and
velocity are used to assign a consistent set of values for both k and omega
based on a freestream turbulence intensity value. The default value is 0.02,
which corresponds to a 2 percent  turbulence intensity. This is the default form
of limiting for the turbulence variables, but this form can be overridden via
specification of {\tt kFreestream} and {\tt omegaFreestream}.

\item{\tt speciesEquationOptions}

This fact is used to specify solution parameters for the species equations,
and is required for all cases where species transport has been activated
(the number of species in the .mdl file is greater than one). Three options
must be specified as follows:
  \begin{enumerate}
     \item{\bf linearSolver} Choose \emph{SGS} for the symmetric Gauss-Seidel
solver.
     \item{\bf relaxationFactor} The relaxation factor used in the non-linear
iterative process. This option is not used by 'stream-piso'.
     \item{\bf maxIterations} The maximum number of iterations to perform in
the linear equation solver.
  \end{enumerate}

\item{\tt plot\_output}

This fact is used to specify optional variables that are to be written to the
\textbf{caseName/output} directory for post-processing. By default,
density, velocity and pressure are always written out. For compressible flows,
temperature is always written out. For multi-species flows, the species
mass fractions are always written out. Optional variables are written
by including them in a comma-separated list as in the following example:
\begin{verbatim}
  plot_output: h,laminarViscosity,viscosityRatio,vResidual,
    pResidual,hResidual,kResidual,omegaResidual,lDES,vAbs
\end{verbatim}

The above list contains all possible variables that can be optionally output.
The user need only include variables of interest. The variable viscosityRatio
represents the ratio of the eddy viscosity to laminar viscosity. The variable
lDES represents the DES length scale. Values greater than one indicate that DES is
active. The variable vAbs represents the absolute fluid velocity, and is used
for turbomachinery problems to output a consistent velocity for all parts of
the domain (The variable v that is automatically output is the velcity of the
fluid relative to the reference frame of the cell, which may vary from cell to
cell).

\item{\tt probe}

This fact is used to specify locations for monitoring solution values. The
format for this variable is as follows:
\begin{verbatim}
probe: <probe0=[x0,y0,z0],probe1=[x1,y1,z1],probe2=[x2,y2,z2],
  ...>
\end{verbatim}

Any number of probes may be used. For
each probe there is a corresponding data file (e.g. probe0.dat) containing the
monitored solution values for each time-step (in time-marching mode) or each
iteration (in steady-state mode). The order of the variables written on each
line of the file is as follows: time-step/iteration number, solution time,
density, vX, vY, vZ, pressure, temperature, total enthalpy, species mass
fractions, probe position and probe distance. The solution time variable has
meaning only in time-marching mode. Species mass fractions are only written to
file for cases in which a mixture material is used. \bf Note: Values written
for a probe correspond to the solution at the cell closest to the probe
location. The probe distance variable indicates the actual distance between
the probe and the closest cell center.

\end{enumerate}

\appendix

\section{Rigid-Body Grid Motion} \label{appendix:rigidBody}

For flows about objects that are moving in a rigid-body manner (such as pluging
and pitching airfoils) one can use the rigid-body grid motion fact to specify
the displacement of the object with time. In this type of simulation, the grid
is considered to be attached to the object, and every node of the grid moves in
a rigid-body manner. This precludes the use of this fact for problems in which
multiple objects are moving in a rigid-body fashion relative to each other. The
motion of the grid nodes consists of two parts, the translational motion and the
rotational motion. The translational displacement of the nodes from the
reference configuration (the grid specified in the .vog file) is defined by
the equation:

\begin{equation}
  s_{translation} = sMag \times \overline{sDir} \times t + tMag \times
    \overline{tDir} \times sin(tFreq \times t + tPhi)
  \label{eq:translation}
\end{equation}

A bulk linear translation in time can be specified using the scalar $sMag$ and
vector $\overline{sDir}$. For the periodic component of the translation, $tMag$
 is a scalar indicating the magnitude of the translation,
$\overline{tDir}$ is a vector indicating an arbitrary direction of translation,
$tFreq$ is the angular frequency of translation and $tPhi$ is a phase lag. The
total angle of rotation of the grid nodes from the reference configuration is
defined by the equation:

\begin{equation}
  \theta = rAlphaBar + rMag \times sin ( rFreq \times t + rPhi )
  \label{eq:rotation}
\end{equation}

where $rAlphaBar$ is the mean angle of rotation, $rMag$ is the magnitude of the
rotation about the mean, $rFreq$ is the frequency of rotation (which can be
different than the translational frequency), and $rPhi$ is a phase lag.

The specification of these parameters in the .vars file using the rigid-body
grid motion fact is done as follows:

\begin{verbatim}

rigidBodyGridMotion: <tDir=[0.0,1.0,0.0],tMag=0.1,tFreq=6.2831852,
  tPhi=0.0,rCenter=[0.0,0.0,0.0],rAxis=[0.0,0.0,1.0],rAlphaBar=0.0,
  rMag=0.0,rFreq=0.0,rPhi=0.0>

\end{verbatim} 

In addition to the translation and rotation parameters mentioned in Eqs.
(\ref{eq:translation}) and (\ref{eq:rotation}), two additional parameters,
$rCenter$, which specifies the center of rotation, and $rAxis$, which defines
an arbitrary axis of rotation through the center of rotation, must be specified
as shown above. \emph{\bf IMPORTANT}: To run cases with rigid-body rotation,
one must request the rigid-body grid motion module as follows when invoking the
executable:
\begin{verbatim}
  mpirun -np 1 stream -load_module gridMotion -q solution
    [-inputs] caseName [debug] [restartIteration]
\end{verbatim}

\section{Boundary Profile File Formats} \label{appendix:boundaryProfile}

Following is the Cartesian boundary profile file format (using temperature as
an example). For an (x-T) profile use 0 for the profile flag. For (y-T) and
(z-T) profiles use 1 and 2 respectively. Data points should be listed in the
file in the order of increasing independent varible (either x, y or z). Note
that any boundary faces located outside the bounds of the profile will be
assigned the value of the nearest endpoint of the profile.

\begin{verbatim}
numDataPoints profileFlag
for(int i=0;i<numDataPoints;++i){
  x[i]  y[i]  z[i]  T[i]
}
\end{verbatim}

Following is the axisymmetric boundary profile file format (using temperature
as an example). Data points should be listed in the file in the order of
increasing radius.

\begin{verbatim}
numDataPoints
for(int i=0;i<numDataPoints;++i){
  r[i]  T[i]
}
\end{verbatim}

\section{Initial Condition Regions} \label{appendix:initialCondition}

See the CHEM users manual.

\end{document}








